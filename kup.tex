\documentclass[a4paper, 18pt]{article}
\usepackage[utf8]{inputenc}
\usepackage{xeCJK}
\setCJKmainfont{Noto Sans CJK TC}
\usepackage[margin=2cm] {geometry}
\usepackage{setspace}
\doublespacing
\usepackage{apacite}
\usepackage{graphicx}
\graphicspath{ {./images/} }
\usepackage{indentfirst}
\setlength{\parindent}{2em}
\usepackage{caption}

\begin{document}
\begin{center}

\vspace*{7cm}
\Huge \textbf{多元學習表現}\\
\rule[10pt]{12cm}{0.05em}\\
\Huge {APCS大學程式設計先修檢測}\\

\vspace{10cm}

\begin{flushright}
\Large 姓名:吳子龍\\
\Large 日期:2022年10月23日\\
\end{flushright}

\end{center}

\newpage
%%%%%%%%%%%%%%%%%%%%%%%%%%%%%%%%%%%%%%%%
\renewcommand{\contentsname}{目錄}
\tableofcontents
%%%%%%%%%%%%%%%%%%%%%%%%%%%%%%%%%%%%%%%%
\newpage

\section{參加動機}
程式一直以來一直是我最興趣也最想要鑽研的科目,雖然在學校沒有程式的課程,但我還是在課外的時候找出時間來自我練習,但我總覺的我必需要有一個檢定來測試自我,所以我就在我的老師的推薦下參加了APCS。

\section{檢定資料}
\subsection{題目題型}
題型分為實作題與觀念題,在一天內應試完畢,實作題考試時間為2小時又30分鐘,觀念題考試時間為2小時。實作題為後測評分,在測試時只有範例測資可以進行測試。\\
實作題分為4題,題目難度也是由簡單到難,越靠前的題目越不需要使用複雜的演算法或資料結構,但在實作上的複雜度也會更加的麻煩。\\
觀念題分為兩節考試,題目難度並沒排序,因此要在一定的時間內解出越多題越有利,也需要練習程式的追蹤,以及在紙上練習計算。
\subsection{題目範圍}
題目以大學程式為基礎設計,包含了兩個部份,演算法及資料結構。\\
\noindent 基本題型包含:
\begin{itemize}
    \item 輸入與輸出
    \item 算術運算、邏輯運算、位元運算
    \item 條件判斷、迴圈
    \item 陣列與結構
    \item 字元、字串
    \item 函數、遞迴
\end{itemize}

\noindent 演算法包含:
\begin{itemize}
    \item 枚舉
    \item 排序
    \item 搜尋
    \item 分治
    \item 貪心
    \item 動態規劃
    \item DFS、BFS
\end{itemize}

\noindent 資料結構包含:
\begin{itemize}
    \item 佇列
    \item 堆疊
    \item 樹狀圖
    \item 圖
\end{itemize}

\subsection{設備}
檢定的程式語言可選用C、C++、Java、Python,以下以C++為例。
\begin{itemize}
    \item 作業系統:Lubuntu Desktop 18.04 (64-bit)
    \item 編輯器:Code::Blocks 17.12
    \item 編譯器:Gcc 7.3.0
    \item C++版本:C++11
\end{itemize}

\section{檢定訓練}
因為在學校沒有相關的課程,所以我使用網路上的資源進行自學,大約訓練了五個月。
\subsection{第一月}
前一個月主要為複習我以前所學的知識,以及大至上的了解考試的內容與方向,並且在這個時間我選擇從我熟惜的Java轉為C++,原因是大份的教學資源皆為C++或Python,但因為在學習上Java比較容易轉換至C++,因此在接上來的訓練皆是使用C++進行。\\
這個月份我學習了DFS、BFS、stack、queue
\subsubsection{DFS}
DFS(深度優先搜尋)用於遍歷一張圖,同一條路徑走到底直到沒有子結點在返回父結點。這是我在這個時間接觸比較難個資料結構。DFS可由stack或遞迴實作,這裡會用到stack與遞迴的互換,其實就是遞迴與迭代的的互換,這也是我第一個卡也比較久的關卡。
\subsubsection{BFS}
BFS(廣度優先搜尋)也是用於遍歷一張圖,與DFS不同的是使用的資料結構不同,BFS使用queue來達到搜尋的效果,會先遍歷所有當前結點的子結點,再將子結點放入queue。
\subsubsection{stack}
stack屬於先進後出的資料結構,就像抽盤子一樣,只會從最前端開始拿,也就是遞迴的原理。
\subsubsection{queue}
queue屬於先進先出的資料結構,像排隊一樣,有一個頭部與尾部,常與while迴圈使用。

\subsection{第二、三月}
第二個月我學習了第一次正式的課程,持續了兩個星期,這兩個星期內容包含了需多簡單和困難的內容,從演算法的時間複雜度與空間複雜度的估計,到複雜的演算法與資料結構,因此我也作了大量的題目進行練習。

課程內容包含:搜尋、枚舉、貪心、分治、數論、動態規劃、基礎資料結構、進階資料結構、圖論、計算幾何。

\subsubsection{搜尋}
搜尋的核心是用找的找到答案,因此資料常常需要排序或使用資料結構。搜尋的課程主要包含二分搜、雙指針、DFS、BFS,主要著重於二分搜的應用。

\paragraph{二分搜}
是非常常用的基本算法之一,也是APCS常常當作題目的算法,主要的核心精神為將有序的資料一次又一次的砍半,實作又可分為迭代、遞迴、倍增,更可以使用C++內建的函式庫。二分搜也常和建表法一起使用,可以在多筆詢問優化時間複雜度,可以達到\(log_2 n\)的時間複雜度。\par
二分搜常見的應用為對答案二分搜,意指針對給出的結果做二分搜,初始的上界與下界可以設為非常大的值。

\paragraph{雙指針}
用於有序的資料,與while迴圈一起使用,當資料比預期大時就讓右指針靠向左指針,反之意然。這是常見的優化時間複雜度的方法,一般可以降低一個冪次的複雜度。

\subsubsection{枚舉}
枚舉是我第一個卡住的演算法,我花了比較多的時間在這上面。枚舉的精神為暴力嘗試所有的答案,雖然時間複雜度高,但有一些方法可以進行優化。枚舉有需多的方法實作,常見的像迴圈枚舉、遞迴枚舉、位元枚舉。應用包含枚舉關鍵點、折半枚舉、回朔法。

\paragraph{迴圈枚舉}
使用像for、while等製作巢狀迴圈,一層枚舉一項,發現偏離目標則break當前的迴圈,但在實作上不比遞迴來的方便。

\paragraph{遞迴枚舉}
遞迴的運用相當的廣,其中枚舉也其中一種,其原理很像DFS,通常也會與回朔法一起使用,其原理為設一個起點並選擇所有可選的選項,可以使用建表紀錄所有的過程,或者是到達要的結果就直接return。

\paragraph{位元枚舉}
使用bitset或是移位符號模仿二進制,只要有'1'就選,'0'則捨棄,因此這個方法只能用於選或不選這種只有兩種狀態的題目,且時間複雜度為\(2^n\),一定要特別的注意題目的資料範圍。

\paragraph{枚舉關鍵點}
並不是所有的情況都需有枚舉整個問題,有些時候只需有枚舉特定的點即可,通常需要使用數學化簡才能發現。

\paragraph{折半枚舉}
假設有一筆資料的資料量高達\(2^{40}\),這樣若直接枚舉將會花費大量的時間,但我們可以一次枚舉一半,若一次枚舉\(2^{20}\),這樣的時間複雜度將會直接砍半,之後我們在將所有的資料排序,這樣即可使用搜尋的方法找出答案。

\paragraph{回朔法}
如果今天在枚舉時答案偏差過大或無法經由後面的計算往回,則這條路徑則可以直接放棄。

\subsubsection{貪心}














\end{document}
